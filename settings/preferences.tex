% настройки polyglossia
\setdefaultlanguage{russian}
\setotherlanguage{english}

% локализация
\graphicspath{ {} }
\addto\captionsrussian{
  \renewcommand{\partname}{Глава}
  \renewcommand{\contentsname}{Содержание}
  \renewcommand{\figurename}{Рисунок}
  \renewcommand{\listingscaption}{Листинг}
}

% основной шрифт документа
\setmainfont{CMU Serif}

% перечень использованных источников
\addbibresource{refs.bib}

% оформление презентации
\usetheme{metropolis}
\usecolortheme{seagull}
\beamertemplatenavigationsymbolsempty

% настройка ссылок и метаданных документа
\hypersetup{unicode=true,colorlinks=true,linkcolor=red,citecolor=green,filecolor=magenta,urlcolor=cyan,        		       
    pdftitle={\docname},   	    
    pdfauthor={\studentname},      
    pdfsubject={\docname},         
    pdfcreator={\studentname}, 	       
    pdfproducer={Overleaf}, 		     
    pdfkeywords={\docname}
}

% настройка подсветки кода и окружения для листингов
\usemintedstyle{colorful}
\newenvironment{code}{\captionsetup{type=listing}}{}

% шрифт для листингов с лигатурами
\setmonofont{[FiraCode-Regular.otf]}[
  Contextuals=Alternate  % Activate the calt feature
]

% путь к каталогу с рисунками
\graphicspath{{fig/}}

% настоящее матожидание
\newcommand{\MExpect}{\mathsf{M}}

% объявили оператор!
\DeclareMathOperator{\sgn}{\mathop{sgn}}