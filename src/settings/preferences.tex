% настройки polyglossia
\setdefaultlanguage{russian}
\setotherlanguage{english}

% локализация
\graphicspath{ {} }
\addto\captionsrussian{
	\renewcommand{\partname}{Глава}
	\renewcommand{\contentsname}{Содержание}
	\renewcommand{\figurename}{Рисунок}
	\renewcommand{\lstlistingname}{Листинг}
}

% основной шрифт документа
\setmainfont{CMU Serif}
\setsansfont{CMU Sans Serif}
\setmonofont{Fira Code}
\newfontfamily{\cyrillicfonttt}{Fira Code}

% перечень использованных источников
\addbibresource{refs.bib}

% оформление презентации
\usetheme{metropolis}
\usecolortheme{seagull}
\beamertemplatenavigationsymbolsempty

% настройка ссылок и метаданных документа
\hypersetup{unicode=true,colorlinks=true,linkcolor=red,citecolor=green,filecolor=magenta,urlcolor=cyan,
	pdftitle={\docname},
	pdfauthor={\studentname},
	pdfsubject={\docname},
	pdfcreator={\studentname},
	pdfproducer={XeLaTeX},
	pdfkeywords={\docname}
}

% настройка подсветки кода и окружения для листингов
\newenvironment{code}{\captionsetup{type=lstlisting}}{}

% путь к каталогу с рисунками
\graphicspath{{fig/}}

% настоящее матожидание
\newcommand{\MExpect}{\mathsf{M}}

% объявили оператор!
\DeclareMathOperator{\sgn}{\mathop{sgn}}

% стиль для листинга
\lstdefinestyle{mystyle}{
	keywordstyle=\color{magenta},
	basicstyle=\ttfamily\footnotesize,
	breakatwhitespace=false,
	breaklines=true,
	captionpos=b,
	keepspaces=true,
	showspaces=false,
	showstringspaces=false,
	showtabs=false,
	tabsize=2
}
\lstset{style=mystyle}
