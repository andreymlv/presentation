\begin{frame}
\titlepage
\end{frame}

\begin{frame}{Заполнение шаблона}
\begin{itemize}
    \item Изменить \textbf{config.tex}: имя студента, название предмета и пр. параметры указаны именно там
    \item Заполнить \textbf{content.tex} - файл, который будет содержать весь текст презентации.
    \item Добавить используемую литературу (если есть) в \textbf{refs.bib}. Для удобного поиска источников можно воспользоваться Google Books. Использованные источники можно указывать с помощью команды \textbf{\\cite\{name\_of\_ref\}}
\end{itemize}
Далее представлены различные примеры.
\end{frame}

\begin{frame}{Обычный слайд текста}
Текст сам центрируется по высоте слайда. Центрирование по горизонтали 
\end{frame}


\begin{frame}{Список}
\begin{itemize}
	\item Элемент списка
	\begin{itemize}
		\item Элемент вложенного списка
	\end{itemize}
	\item Элемент списка
	\begin{enumerate}
		\item Элементы
		\item нумерованного
		\item списка
	\end{enumerate}
\end{itemize}
\end{frame}

\begin{frame}{Блоки}
\begin{exampleblock}{Заголовок блока}
	Текст блока примера. Кстати, ссылка \cite{kingma2014adam}
\end{exampleblock}

\begin{alertblock}{}
	Текст блока предупреждения 
\end{alertblock}

\end{frame}

\begin{frame}{Рисунок}
\begin{figure}[H]
	\includegraphics[width=0.5\textwidth]{fig/sample.png}
	\label{fig:sample}
	\caption{Подпись}
\end{figure}
\center{\url{https://impurepics.com/}}
\end{frame}

\begin{frame}{Листинг}
\begin{code}
    \inputminted[breaklines=true, xleftmargin=1em,linenos, frame=single, framesep=10pt, firstline=1, lastline=7]{haskell}{listings/haskell_code.hs}
    \caption{Да, это опять функциональный код.}
\end{code}
\end{frame}

\begin{frame}{Формулы, сложна}
Спектр (спектральная плотность) $\Phi(f)$ в общем случае представляет собой комплексную функцию: $$\Phi(f)=|\Phi(f)|*e^{i\psi(f)}$$
Модуль этой функции $|\Phi(f)|$ называют спектром амплитуд, а зависимость $\psi(f)$ — спектром фаз.
\end{frame}

\begin{frame}[fragile]{Разбиваем слайд}
Раз.
\pause
Два.
\pause
Ну вы поняли.
\end{frame}

\begin{frame}{Анимация?}
\begin{figure}
\begin{tabular}{c}
Но нужен Adobe Reader. Или что-то другое хорошее. \\
  \animategraphics[loop,controls,width=0.9\textwidth]{1}{fig/sgd/descent-}{0}{7}
\end{tabular}
\end{figure}
\end{frame}

\begin{frame}{Перечень использованных источников}
\printbibliography[title=Этот цвет тоже можно поменять.]
\end{frame}
